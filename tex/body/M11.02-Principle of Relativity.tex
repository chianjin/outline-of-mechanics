\section{相对性原理}\label{sec:11.02}

在牛顿的体系中,在理论上惯性系是由绝对时间和绝对空间
来确定的。牛顿认为,绝对时间和绝对空间是最基本的惯性系。
他曾写道:“绝对的、纯粹的、数学的时间,就其本性来说,均
匀地流逝,而与任何外在的情况无关。”“绝对空间,就其本性
来说,与任何外在情况无关,始终保持着相似和不变。”这就是
说,牛顿认为绝对空间是真正的绝对的静止。按照这种理论定
义,我们就应当问,各个实用惯性系相对于绝对空间的速度是多
少?

上述问题还有另外一种提法,而且是更早的提法。在牛顿之
前,也有绝对空间静止的概念。亚里士多德体系认为,地球是宇
宙的中心,它是绝对静止的。在哥白尼体系中,中心不是地球,
而是太阳,地球围绕太阳转。这还是绝对空间、绝对静止观念。
地心说与日心说之间的争论问题之一是,若地球绕太阳转,其绝
对速度如此之快,为什么我们感觉不到?这是从力学观点上对哥白

% 320.jpg
\clearpage\noindent
尼学说的非常大的非难。

伽利略对上述问题给了一个彻底的回答,他说:

\begin{quoting}
    “把你和一些朋友关在一条大船的甲板下的主舱里,
让你们带着几只苍蝇、蝴蝶和其他小飞虫,舱内放一只
大碗,真中有几条鱼。然后,挂上个水瓶,让水一滴一
滴地滴到下面的一个宽口罐厘。船停着不动时,你留神
观察,小虫都以等速向舱内各方向飞行,鱼向各方向随
便游动,水滴滴进下面的罐中,你把任何东西扔给你的
朋友时,只要距离相等,向这一方向不必比另一方向用
更多的力。你双脚齐跳,无论向哪个方向,跳过的距离
都相等。当你仔细地观察这些事情之后,再使船以任何
速度前进,只要运动是匀速,也不忽左忽右地摆动,你
将发现,所有上述现象丝弯没有变化。你也无法从其中
任何一个现象里确定,船是在运动还是停着不动。即使
船运动得相当快,在跳跃时,他也将和以前一样,在船
底板上跳过相同的距离,你跳向船尾也不会比跳向船头
来得远些。虽然你跳到空中时,脚下的船底板向着你跳
的相反方向移动。你把不论什么东西扔给你的同伴时,
不论他是在船头还是在船尾,只要你自己站在对面,你
也并不需要用更多的力。水滴将象先前一样,滴进下面的
罐子,一滴也不会滴向船尾。虽然水滴在空中时,船已
行驶了许多柞\sbfootnote{“柞”为大指尖到小指尖伸开之长,通常为九英寸,是古代一种长度单位。}。鱼在水中游向水碗前部所用的力并不
比游向水碗后部来得大;它们一样悠闲地游向放在水碗
边缘任何地方的食饵。最后,蝴蝶和苍蝇继续随便地到
处飞行,它们也决不会向船尾集中,并不因为它们可能
长时间留在空中,脱离开了船的运动,为赶上船的运动
% 321.jpg
\clearpage\noindent
而显出累的样子。”\sbfootnote{类似伽利略哪种原始形式的相对性原理思想,中国古代也曾有过。在公元前
一世纪成书的《尚书纬·考灵曜》中写着:“地恒动不止,而人不知,如人在大舟
中,闭窗而坐,舟行而人不觉也。”当然,这种原始的思想,在当时不可能形成完
整的科学体系。}
\end{quoting}

总之,伽利略认为,从船中发生的任何一种现象,你是无法
判断船究竟是在运动还是停着不动。在地球上你并不能感到地球
的运动。

用牛顿力学的语言来说,大船就是一种惯性系。在一个惯性
系中所能看到的种种现象,在另一个惯性系中必定也能无任何差
别地看到。也就是所有惯性系都是平权的、等价的。因此,我们
不可能判断哪个惯性系是处于绝对静止状态,哪个是有绝对运动
的。这个论断称为伽利略相对性原理。

伽利略相对性原理不仅从根本上解决了对哥白尼学说的力学
非难,而且在惯性运动范围内否定了绝对空间观念。牛顿对伽利
略的论断是十分了解的,他知道实际上无法测出绝对速度,亦即
在各惯性系中间,无法断定哪一个相应于绝对空间。所以,牛顿
在讨论绝对空间时,也并未提出测量绝对速度,而只强调了测量
绝对的加速度。对后一点,我们将在下章中讨论。

牛顿的绝对空间观念虽然不合于伽利略相对性原理,但是牛
顿力学规律却是与相对性原理相治的。相对性原理要求,所有惯
性系是平权的。那么,在所有的惯性系中,动力学规律的形式应当
是相同的。现在我们证明,牛顿第二定律是符合这个要求的。
有两个惯性系$ K $及$ K' $,$ K' $相对于$ K $以速度$ \vec{u} $运动。在$ K $系中,
牛顿动力学规律是
\begin{equation}\label{eqn:11.02.01}
    \vec{F} = m \vec{a}
\end{equation}
$ K $及$ K' $系之间的时空遵从伽利略变换关系,即
\begin{equation}\label{eqn:11.02.02}
    \vec{r} ^ { \prime } = \vec{r} - \vec{u} t - \vec{d} _ { 0 }
\end{equation}
% 322.jpg
\begin{equation*}
    t ^ { \prime } = t + t _t { 0 }
\end{equation*}
其中各量的物理意义已在第二章第五节中讨论过,由此容易推出
\begin{equation*}
    \frac { \dif ^ { 2 } r ^ { \prime } } { \dif t ^ { \prime 2 } } = \frac { \dif ^ { 2 } \vec{r} } { \dif t ^ { 2 } }
\end{equation*}
这个式子表示,一个物体相对于$ K $系的加速度与相对$ K' $系的是一
样的,即
\begin{equation}\label{eqn:11.02.03}
    \vec{a} ^ { \prime } = \vec{a}
\end{equation}

另外,两物体间的作用力常常只取决于物体之间的距离,而
在伽利略变换下,距离是不变量,所以,同一相互作用力相对于
$ K $与相对于$ K' $是一样的,即
\begin{equation}\label{eqn:11.02.04}
    \vec{F} ^ { \prime } = \vec{F}
\end{equation}

再则,在牛顿体系中也假定质量是不变的,同一物体,在任
何惯性系中都具有相同的质量,即
\begin{equation}\label{eqn:11.02.05}
    m ^ { \prime } = m
\end{equation}

这样,将式\eqref{eqn:11.02.03}~\eqref{eqn:11.02.05}代入式\eqref{eqn:11.02.01},就得到
\begin{equation}\label{eqn:11.02.06}
    \vec{F} ^ { \prime } = m ^ { \prime } \vec{a} ^ { \prime }
\end{equation}
可见,在$ K' $系中牛顿动力学方程仍与\eqref{eqn:11.02.01}形式完全一样。这
就完全证明了牛顿动力学规律与相对性原理是相洽的。