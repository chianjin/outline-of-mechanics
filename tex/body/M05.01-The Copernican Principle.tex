\section{哥白尼原理}\label{sec:05.01}

牛顿在处理行星运动的时候,只考虑了太阳对行星的引力作
用,但却完全忽略了其他恒星等天体对行星的作用。这是一个非
常大胆的假设,但他却获得了正确的结果。在牛顿之后,研究太
阳系中或地球附近的力学问题时,实质上也都暗含地采用了上述
假设,完全忽略了其他天体的作用。

因此,我们要问:为什么可以忽略其他星体的作用呢?为什
么完全忽略其他星体的作用之后,还能得到许多正确的结果呢?

理由之一是引力质量与惯性质量之比是普适常数。由此即可
忽略其他恒星对太阳-行星体系的作用,就象在研究地球-月亮体
系的运动时,在相当精度范围内,可以先不计太阳的作用。然而,
这个理由并不能使我们完全忽略恒星的作用。特别是体系越大越
不能完全忽略。太阳-行星体系比地球-月亮体系大得多,所以这
个理由是不充分的。

理由之二是恒里距离太远,所以它对行星的引力很小,可以
忽略。的确,最近的恒星距我们约为4.3光年(一光年等于光在一
年中运行的距离,约等于$ 9.4605 \times 10 ^ { 15 } $米),其他天体就更加遥远,
每颗恒星对行星的作用是非常小的。但是,如果宇宙是无限的,
恒星的数目也无限,则无限恒星总和的力却可能极大极大。因此,
% 156.jpg
只当宇宙有限,星体数目有限时,这个理由才成立。

这种情况迫使我们采用第三个理由,即认为星体的分布相对
于我们是对称的。如果在每个与我们距离相等的球壳状天区内星
体分布是均匀的,则由上章最后的讨论,这种球壳对太阳-行星
体系的作用严格为零。

这种模型的确很理想,但它有一个严重的观念上的困难。因
为在这种模型里,似乎太阳或地球又成为宇宙的中心,这是我们
难于接受的。从哥白尼以来,一个最重要的科学成果是认识到地
球不是宇宙的中心,太阳也不是宇宙的中心。

上述种种分析告诉我们,虽然物理学往往着眼于研究局部的
现象,但是并不能完全地回避整个宇宙的问题。因为整体宇宙性
质会影响到局部的现象。我们要想建立一个自治的局部现象的理
论,也必须对宇宙整体性质给以说明。

本章的目的就是要建立一个恰当的宇宙模型,使它能够说明
本节一开始提出的问题,同时又不与宇宙不存在中心这一观念相
矛盾。因此,我们的讨论首先就必须坚持以下的出发点:

\begin{quoting}
  在宇宙中没有特殊的位置,每一个观察者看到的现象都
  是一样的。
\end{quoting}

这个陈述常被称为哥白尼原理。所谓“没有特殊位置”,意
即没有中心,所谓“每个观察者看到的都一样”,意即宇宙间各
点是平权的。应该说明,每个观察者看到的现象,并非指所有的现
象,而是大尺度上的平均现象。从小尺度来看宇宙各处都有许多
差异,但从大尺度的平均来看,的确是差异较小的。